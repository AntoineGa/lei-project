\chapter{Conclusões}


\section{Formações Adicionais}

De forma a conseguir adquirir novos conhecimentos e aprofundar conhecimentos já existentes, foram realizadas algumas formações adicionais durante os tempos livres (como fins de semana, por exemplo). A lista que se segue apresenta alguns dos projetos realizados durante estas formações, bem como as respetivas ligações para os mesmos.


\begin{minipage}[Ht]{0.45\textwidth}
	\begin{itemize}
		\item \textbf{Next Level Week 04}
			\begin{itemize}
				\item \href{http://nextlevelweek.com/}{Informações};
				\item \textbf{Repositório:} \href{https://github.com/TutoDS/nlw04-react}{GitHub};
				\item \textbf{\textit{Deploy}:} \href{https://move-it-tutods.vercel.app}{Vercel};
			\end{itemize}

		\item \textbf{\textit{ReactJS Challenge~\textemdash~Slack Clone}}
			\begin{itemize}
				\item \href{https://www.youtube.com/channel/UCqrILQNl5Ed9Dz6CGMyvMTQ}{Canal do YouTube}
				\item \textbf{Repositório:} \href{https://github.com/TutoDS/reactjs-slack-clone-challenge}{GitHub};
				\item \textbf{\textit{Deploy}:} \href{https://slack-clone-challenge-c35ca.web.app/}{Firebase};
			\end{itemize}

			\item \textbf{Twitter \textit{UI Clone}}
			\begin{itemize}
				\item \href{https://www.youtube.com/watch?v=K-8z_4xvT3o}{Vídeo}
				\item \textbf{Repositório:} \href{https://github.com/TutoDS/twitter-ui-clone}{GitHub};
				\item \textbf{\textit{Deploy}:} \href{https://twitter-clone-tutods.netlify.app/}{Netlify}
			\end{itemize}
	\end{itemize}
\end{minipage}
\begin{minipage}[Ht]{0.45\textwidth}
	\begin{itemize}
		\item \textbf{Next Level Week 05}
			\begin{itemize}
				\item \href{http://nextlevelweek.com/}{Informações};
				\item \textbf{Repositório:} \href{https://github.com/TutoDS/nlw05-react}{GitHub};
				\item \textbf{\textit{Deploy}:} \href{https://podcastr-tutods.vercel.app/}{Vercel}
			\end{itemize}

		\item \textbf{LinkedIn \textit{UI Clone}}
			\begin{itemize}
				\item \href{https://www.youtube.com/watch?v=xP3cxbDUtrc}{Vídeo}
				\item \textbf{Repositório:} \href{https://github.com/TutoDS/reactjs-linkedin-clone}{GitHub}
				\item Por Concluir
			\end{itemize}

		\item \textbf{TypeGraphQL} - Code/drop \#74
			\begin{itemize}
				\item \href{https://www.youtube.com/watch?v=qMc5A5-Ktuw}{Vídeo}
				\item \textbf{Repositório:} \href{https://github.com/TutoDS/typegraphql-code-drops-74}{GitHub}
				\item \textbf{Temas Principais:} GraphQL e TypeGraphQL
			\end{itemize}
	\end{itemize}
\end{minipage}

\vspace{10pt}

Nos projetos apresentados os principais conhecimentos aplicados foram relacionados com a biblioteca \textbf{React}, mais precisamente na utilização de contextos, o uso de \textit{Styled Components} para realizar todo o \textit{layout} da aplicação (em vez de \textbf{CSS} ou \textbf{Sass}) e o uso de dois temas, no caso \textit{dark} e \textit{light mode}.

\section{Trabalhos Futuros}

\begin{itemize}
	\item Testes com JEST - para \textit{unit test}
	\item Testes com Cypress - para \textit{end-to-end}
	\item Biblioteca de componentes
\end{itemize}