\section{Organização do Documento}

Este documento encontra-se organizado em vários capítulos, de forma a facilitar a leitura do mesmo. Desta forma é possível encontrar os seguintes capítulos:

\begin{itemize}
	\item \textbf{Capítulo 1 \textemdash~ Contextualização e Motivação:} no primeiro capítulo é possível encontrar uma breve introdução ao projeto, sendo apresentados os principais objetivos e como se encontra organizado este documento, apresentando os principais tópicos de cada capítulo;

	\item \textbf{Capítulo 2 \textemdash~ Fundamentação Teórica:} neste capítulo são abordadas as tecnologias e ferramentas utilizadas no decorrer do projeto, bem como a metodologia utilizada e informações relativas ao controlo de versões;

	\item \textbf{Capítulo 3 \textemdash~ Visão geral do projeto:} na visão geral do projeto é possível conhecer melhor o projeto, conhecer os objetivos, dependências e quem são os utilizadores do projeto;

	\item \textbf{Capítulo 4 \textemdash~ Design \& Implementação da solução:} no que toca ao design e Implementação da solução é possível encontrar essencialmente diagramas referentes ao mesmo, começando pelo diagrama da arquitetura conceptual, passando aos diagramas de sequência onde é apresentado de forma geral o fluxo da aplicação em determinados ecrãs do mesmo. No tópico referente aos diagramas de sequência são também apresentados os componentes \textbf{React} com mais importância, apresentando as partes importantes da sua Implementação. Por fim, é apresentada a estrutura de pastas utilizada em ambas as componentes do projeto (o \textit{backoffice} e \textit{frontoffice});

	\item \textbf{Capítulo 5 \textemdash~ Resultados:} como o nome deste capítulo indica, serão apresentados os resultados obtidos referentes ao desenvolvimento do projeto, bem como algumas reflexões sobre o mesmo;

	\item  \textbf{Capítulo 6 \textemdash~ Conclusões:} neste capítulo é possível encontrar uma conclusão global referente ao projeto, algumas das formações realizadas de forma a conseguir melhorar a prestação durante a realização do projeto, bem como os trabalhos a serem implementados futuramente.
\end{itemize}

Importante referir que em anexos é possível encontrar informações adicionais relativas às tecnologias utilizadas, bem como alguns guias referentes às mesmas.