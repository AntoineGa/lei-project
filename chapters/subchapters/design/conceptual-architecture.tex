\section{Arquitetura Conceptual}

O projeto em questão encontra-se composto por duas componentes, a parte de \textit{back-end} e o \textit{front-end}. A imagem que se segue representa a arquitetura, bem como a comunicação entre ambas as componentes.

\figureFrame{1}{architecture.png}{Arquitetura Conecptual do projeto}

Como é possível analisar do lado do servidor (\textit{back-end}) é possível encontrar uma \textbf{\acrshortplUnder{api}} composta por:

\begin{itemize}
	\item \textbf{GraphQL};
	\item \textbf{Apollo Server};
	\item \textbf{TypeORM};
	\item \textbf{KoaJS}\footnote{Refêrencia Bibliográfica sobre \textbf{KoaJS}: \cite{expressVsKoa,introkoa}} {\small(não representado no diagrama)};
	\item \textbf{PostgreSQL};
\end{itemize}

Já do lado do cliente (\textit{front-end}), é possível encontrar como base do projeto a biblioteca \textbf{React}, utilizada nas duas vertentes do lado do cliente, o \textit{backoffice}, direcionado ao administrador e \textit{personal trainers} da plataforma e no \textit{frontoffice}, destinada aos atletas. No lado do cliente é também usado o \textbf{Apollo Client} para realizar \textit{queries} no \textbf{GraphQL}.

Ambas as compoentens (\textit{front-end} e \textit{back-end}) comunicam com serviços externos, no caso o \textbf{Stripe} para pagamentos e o \textbf{Strava} para informações relacionadas com o atleta (como corridas realizadas, entre outras).
