\subsection{Atleta \textemdash~criação de conta}
\label{registerSequenceDiagram}

A criação da conta de um atleta encontra-se presente na vertente de \textit{frontoffice}, sendo este não só executado num passo. Assim sendo, ao longo deste tópico serão apresentados os diagramas de sequência para cada um destes passos.

\figureFrame{1}{sequence/register.png}{Diagrama de Sequência para a criação de conta de um atleta}

O primeiro passo é a criação de conta básica, onde o atleta necessita de preencher os campos:

\begin{itemize}
	\item Nome;
	\item Apelido;
	\item E-mail;
	\item \textit{Password}.
\end{itemize}

Após o preenchimento destes campos e, a aceitação dos \textbf{Termos \& Condições}, é enviado um pedido à \textbf{\glsShortUnder{api}}, onde por sua vez realiza determinadas validações.

\figureFrame{.75}{register-frontoffice.png}{Mockup do ecrã de criação de conta \textemdash~frontoffice}

No caso, a \textbf{\glsShortUnder{api}} realiza uma consulta à base de dados para verificar se os dados enviados já existem, se existirem é devolvida uma mensagem ao cliente a informar que ocorreu um erro. No figura que se segue são apresentados dois erros, um deles informando que o e-mail já se encontra registado.

\figureFrame{.75}{error-on-register-frontoffice.png}{Mockup do ecrã de criação de conta com dados inválidos \textemdash~frontoffice}

No caso de não ocorrer nenhum erro e, a \textbf{\glsShortUnder{api}} devolver uma mensagem de sucesso, o atleta é redirecionado para o último passo do registo. Neste último passo são pedidos ao atleta outros dados pessoais como data de nascimento, contacto telefónico, morada, dados biométricos (altura, peso, frequência cardíaca máxima registada e frequência cardíaca em repouso) e o objetivo pessoal pretendido.

\figureFrame{.75}{complete-register-frontoffice.png}{Mockup de informações de registo \textemdash~frontoffice}

Neste último ecrã de criação de conta é possível encontrar alguns componentes criados, sendo encontrados essencialmente neste ecrã, são os \textit{inputs} de carregamento de foto de perfil e imagem de fundo, bem como a \textit{tooltip} que aparece ao colocar o rato no icone de questão.

\begin{minipage}{0.45\textwidth}
	\figureFrame{1}{file-inputs.png}{Inputs de carregamento de imagens}
\end{minipage}
\begin{minipage}{0.45\textwidth}
	\figureFrame{1}{tooltip.png}{Componente \textbf{Tooltip}}
\end{minipage}

O componente \textbf{Tooltip} tem como objetivo apresentar informação adicional que não esteja presente no ecrã, no caso da imagem apresentada anteriormente, indica como preencher os dados biométricos do atleta. Já os \textit{inputs} destinados ao carregamento da imagem de perfil e imagem de fundo estes são posteriormente utilizadas na página do atleta, tal como é apresentado de seguida.

\figureFrame{1}{athlete-profile-images.png}{Imagens carregadas para o perfil do atleta \textemdash~frontoffice}