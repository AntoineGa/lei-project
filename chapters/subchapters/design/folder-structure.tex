\section{Estrutura de Pastas}

A estrutura de pastas em ambas as componentes do projeto (\textit{frontoffice} e \textit{backoffice}), existindo apenas ficheiros diferentes. A figura que se segue representa a estrutura geral utilizada.

\begin{figure}[H]
	\centering
	\begin{forest}
		for tree={
			folder,
			font=\ttfamily,
			grow'=0
		}
		[{Raíz do Projeto}
			[...]
			[src
				[assets
					[media]
					[styles]
				]
				[components
					[inputs
						[TextInput
							[TextInput.tsx]
							[TextInput.module.sass]
						]
					]
				]
				[controllers
					[App
						[App.tsx]
						[App.module.sass]
					]
					[...]
				]
				[hooks]
				[interfaces]
				[utils]
				[index.tsx]
				[index.sass]
			]
		]
	\end{forest}

	\caption{\textbf{React} \textemdash~estrutura de pastas utilizada}
\end{figure}


Como é possível analisar a pasta principal do projeto é a para \verb|src/|, sendo esta que contêm desde os componentes, estilos, rotas, etc.. Cada componente ou \textit{controller} é constituído (na maioria dos casos) por um ficheiro \verb|tsx| e outro \verb|sass|, sendo o ficheiro com extensão \verb|module.sass| reponsável pelo estilo de determinado componente ou \textit{controller}.

No caso dos componentes, sempre que existe mais do que um componente de uma mesma ``categoria'' (como por exemplo os \textit{inputs}) é criada uma pasta principal para todos os componentes dessa ``categoria''.