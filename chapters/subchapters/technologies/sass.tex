\section{Sass}

\begin{minipage}{.3\textwidth}
	\figureFrame{.45}{sass}{\textbf{Sass} \textemdash~logo}
\end{minipage}
\begin{minipage}{.7\textwidth}
	\minipagerestore
	O \textbf{\glsShortUnder{sass}}, ou \underline{\glsxtrlong{sass}} é um \textit{preprocessor} de \textbf{\glsShortUnder{css}}, possuíndo duas variantes:

	\begin{itemize}
		\item \texttt{.sass} \textemdash~não necessita de \texttt{;} nem \verb|{}|, apenas que o código esteja corretamente indentado;
		\item \texttt{.scss} \textemdash~esta variante necessita de \texttt{;} e \verb|{}|, bem como a correta indentação do código.
	\end{itemize}
\end{minipage}

Durante a realização deste projeto será utilizada a variante sem \texttt{;} e \verb|{}|, sendo apresentados os principais detalhes da mesma.

\figureFrame{1}{sass-to-css.png}{Ficheiro \texttt{.sass} compilado para um ficheiro \texttt{.css}}

A imagem anterior representa a transformação que é realizada após a compilação de um código \textbf{\glsShortUnder{sass}} (\texttt{.sass}), onde é possível analisar a declaração de uma variável (\verb|$darkColor|), bem como o seu uso. Além disso, como é possível analisar, o \textbf{\glsShortUnder{sass}} permite o uso de \textit{nesting} % <== EXPLICAR MELHOR!!!!


As principais vantagens do uso \textbf{\glsShortUnder{sass}} passam desde a criação de \textit{mixins}, funções e ainda herança de estilos. Os excertos apresentados de seguida demonstram estas mesmas vantagens.

\begin{longlisting}
	\begin{minted}{sass}
		=flex-settings($direction: row)
			display: flex
			flex-direction: $direction

		.my-row
			+flex-settings()

		.my-column-row
			+flex-settings(column)
	\end{minted}
	\caption{Definição e uso de \textit{mixins} no \textbf{Sass}}
\end{longlisting}