\subsection{Sass}

\begin{minipage}{.3\textwidth}
	\figureFrame{.45}{sass}{\textbf{Sass} \textemdash~logo}
\end{minipage}
\begin{minipage}{.7\textwidth}
	\minipagerestore
	O \textbf{\glsShortUnder{sass}}, ou \underline{\glsxtrlong{sass}} é um \textit{preprocessor} de \textbf{\glsShortUnder{css}}, possuíndo duas variantes:

	\begin{itemize}
		\item \texttt{.sass} \textemdash~não necessita de \texttt{;} nem \verb|{}|, apenas que o código esteja corretamente indentado;
		\item \texttt{.scss} \textemdash~esta variante necessita de \texttt{;} e \verb|{}|, bem como a correta indentação do código.
	\end{itemize}
\end{minipage}

Durante a realização deste projeto será utilizada a variante sem \texttt{;} e \verb|{}|, sendo apresentados os principais detalhes da mesma.

\figureFrame{1}{sass-to-css.png}{Ficheiro \texttt{.sass} compilado para um ficheiro \texttt{.css}}

A imagem anterior representa a transformação que é realizada após a compilação de um código \textbf{\glsShortUnder{sass}} (\texttt{.sass}), onde é possível analisar a declaração de uma variável (\verb|$darkColor|), bem como o seu uso. Além disso, como é possível analisar, o \textbf{\glsShortUnder{sass}} permite o uso de \textit{nesting}, ou seja, ... TODO

Os pontos que se seguem demonstram algumas das vantagens em utilizar \textbf{\glsShortUnder{sass}}, demonstrando com exemplos práticos.

% ==> MIXINS <== %
\subsection{\textit{Mixins}}

As \textit{mixins} no \textbf{\glsShortUnder{sass}} permitem reutilizar estilo, poupando tempo e aplicando o conceito de \textbf{DRY}, ou seja, \textit{Don't Repeat Yourself}. No excerto de código que se segue é possível analisar a definição de uma \textit{mixin}, bem como a utilização da mesma.

\begin{longlisting}
	\inputminted[highlightlines={6,9},highlightcolor=yellow!25]{sass}{code/sass/mixins.sass}
	\caption{Definição e uso de \textit{mixins} no \textbf{Sass}}
\end{longlisting}

Tal como é possível analisar no excerto de código anterior, as \textit{mixins} podem receber parâmetros, sendo que estes parâmetros podem assumir um valor por defeito. Ou seja, a \textit{mixin} \verb|flex-settings| pode receber ou não a direção (\verb|direction|), sendo o valor por defeito \texttt{row}.

Na linha 6 e 9 é possível analisar a utilização desta \textit{mixin}, bem como a alteração do valor do parâmetro \verb|direction| para \verb|column|.

\begin{mybox}{estg}{Nota}
	A declaração de uma \textit{mixin} em \textbf{\glsShortUnder{sass}} é realizada através do símbolo \verb|=|, seguido do nome da \textit{mixin} e entre parêntesis o(s) parâmetro(s). O uso desta é realizado através do símbolo \verb|+|, seguido do nome e parâmetros caso possua.
	
	Caso seja usada a variante \verb|.scss|, a declaração de \textit{mixins} é feita através de \verb|@mixin| e o seu uso através de \verb|@include|.
\end{mybox}

% ==> EXTENDS <== %
\subsubsection{Herança}

No \textbf{\glsShortUnder{sass}} também é possível realizar herança, nesta caso herança de estilos. O excerto de código\footnote{Retirado da \href{https://sass-lang.com/documentation/at-rules/extend}{\textbf{documentação oficial}}.} que é apresentado de seguida demonstra a utilização da herança no \textbf{\glsShortUnder{sass}}.

\begin{longlisting}
	\inputminted{sass}{code/sass/extends.sass}
	\caption{Demonstração de herança no \textbf{Sass}}
\end{longlisting}

No excerto de código abaixo é possível analisar o resultado final após este ser compilado para um ficheiro \textbf{\glsShortUnder{css}}.

\begin{longlisting}
\begin{minted}{css}
	.error,
	.error--serious {
		border: 1px #f00;
		background-color: #fdd;
	}

	.error--serious {
		border-width: 3px;
	}
\end{minted}
\caption{Código \textbf{CSS} resultante da compilação do excerto de código anterior}
\end{longlisting}

Além da herança através de \textit{class's} é possível recorrer a \textit{placeholders}. \textit{Placeholders} funcionam como uma \textit{class}, porém começa com \verb|%| e não são incluídos no código \textbf{\glsShortUnder{css}} resultante.

\begin{longlisting}
	\inputminted{sass}{code/sass/placeholders.sass}
	\caption{Demonstração de \textit{placeholders} em \textbf{Sass}\footnote{Excerto de código retirado da \textbf{\href{https://sass-lang.com/documentation/style-rules/placeholder-selectors}{documentação oficial}}}}
\end{longlisting}

Sendo que após este código ser compilado, o \textit{placeholder} apresentado não estará no código \textbf{\glsShortUnder{css}} resultante, tal como é possível analisar abaixo.

\begin{longlisting}
	\begin{minted}{css}
		.action-buttons, .reset-buttons {
			box-sizing: border-box;
			border-top: 1px rgba(0, 0, 0, 0.12) solid;
			padding: 16px 0;
			width: 100%;
		}

		.action-buttons:hover, .reset-buttons:hover {
			border: 2px rgba(0, 0, 0, 0.5) solid;
		}

		.action-buttons {
			color: #4285f4;
		}

		.reset-buttons {
			color: #cddc39;
		}
	\end{minted}
	\caption{Código \textbf{CSS} resultante da compilação do excerto de código com \textit{placeholder}}
\end{longlisting}

% ==> VARIABLES <== %
\subsection{Variáveis}

No \textbf{\glsShortUnder{sass}} é possível declarar variáveis recorrendo ao símbolo \verb|$| seguido do nome pretendido. Nos excertos de código que se seguem é possível analisar a declaração de variáveis, o seu uso e qual o resultado após este ser compilado para \textbf{\glsShortUnder{css}}.

\begin{longlisting}
	\begin{minted}[highlightlines={5,6},highlightcolor=yellow!25]{sass}
		$padding: 10px 20px
		$defaultColor: #ca4d24

		.alert
			background-color: $defaultColor
			padding: $padding
	\end{minted}
	\caption{Utilização de variáveis em \textbf{Sass}}
\end{longlisting}

No \textbf{\glsShortUnder{css}} estas variáveis não são visíveis, uma vez que o valor destas serão apresentadas diretamente na linha da sua utilização, ou seja:

\begin{longlisting}
	\begin{minted}[highlightlines={2,3},highlightcolor=yellow!25]{css}
		.alert {
			background-color: #ca4d24;
			padding: 10px 20px;
		}
	\end{minted}
	\caption{Código \textbf{CSS} resultante do excerto de código com variáveis em \textbf{Sass}}
\end{longlisting}

Porém em \textbf{\glsShortUnder{css}} também é possível utilizar variáveis, porém estas são definidas recorrendo a \verb|:root {}|. No excerto de código que se segue é apresentado um exemplo de variáveis em \textbf{\glsShortUnder{css}}.

\begin{longlisting}
	\begin{minted}{css}
		:root {
			--padding: 10px 20px;
			--default-color: #ca4d24
		}

		.alert {
			padding: var(--padding);
			background-color: var(--default-color);
		}
	\end{minted}
	\caption{Declaração e uso de variáveis em \textbf{CSS}}
\end{longlisting}