\section{React}

\begin{minipage}[t]{.3\textwidth}
	\figureFrame{1}{react.png}{\textbf{React} \textemdash~logo}
\end{minipage}
\begin{minipage}[t]{.7\textwidth}
	\minipagerestore
	Existem quem considere que o \textbf{React} é uma \textit{framework} de \textbf{JavaScript}, porém e, ao mesmo tempo, há quem a considere como uma biblioteca de \textbf{JavaScript} baseada em componentes, sendo este o termo correto.

	Os principais objetivos desta biblioteca são essencialmente:

	\begin{itemize}
		\item Fácil Aprendizagem;
		\item Rápidez;
		\item Escalável.
	\end{itemize}
\end{minipage}

\vspace{0.2cm}

Importante referir que em 2020, segundo o \href{https://insights.stackoverflow.com/survey/2020#technology-most-loved-dreaded-and-wanted-web-frameworks-loved2}{StackOverflow}, o \textbf{React} ficou em segundo lugar nas \textit{frameworks} preferidas dos programadores e em primeiro lugar nas mais procuradas.

Em \underline{\hyperref[reactApp]{anexo}} é possível encontrar todas as instruções relativas à criação de um projeto em \textbf{React}.

% ==> Project folder structure <== %

\subsection{Estrutura de Pastas}

A estrutura de pastas para um projeto \textbf{React} pode variar de projeto para projeto, ou da forma como o programador prefere organizar os mais diversos ficheiros do projeto. Porém e, tal como é possível analisar na figura que se segue, é comum encontrar a seguinte estrutura de pastas.

\begin{figure}
\centering
\begin{forest}
	for tree={
	  folder,
	  font=\ttfamily,
	  grow'=0
	}
	[{Raíz do Projeto}
	   [{public}]
	   [src
		  [assets]
		  [components]
		  [controllers]
		  [hooks]
	   ]
	]
\end{forest}
\caption{\textbf{React} \textemdash~possível estrutura de pastas}
\end{figure}

Importante referir que a pasta \texttt{src/} será a pasta principal, uma vez irá conter todos os componentes, \textit{assets} e outros ficheiros importantes para o projeto.

É importante referir que seguindo o método de criação do projeto \textbf{React} com o \texttt{create-react-app}, a estrutura de pastas e os ficheiros criados inicialmente será a seguinte:

\begin{figure}[h!]
\centering
\begin{forest}
	for tree={
		folder,
		font=\ttfamily,
		grow'=0
	}
	[{Raíz do Projeto}
		[public
			[index.html]
			[favicon.ico]
		]
		[src
			[App.css]
			[App.js]
			[App.test.js]
			[index.css]
			[index.js]
			[logo.svg]
		]
	]
\end{forest}
\caption{\textbf{React} \textemdash~estrutura de pastas e ficheiros gerados pelo \texttt{create-react-app}}
\end{figure}

\vspace{0.25cm}
\begin{mybox}{estg}{Nota}
	É importante relembrar que consoante o uso de \textbf{TypeScript} ou \textbf{JavaScript} será possível encontrar ficheiros \texttt{.tsx} ou \texttt{.ts}, \texttt{.jsx} ou \texttt{js}.

%	Além da extensão dos ficheiros será possível encontrar na raíz do projeto um ficheiro de configuração, podendo ser:
%	\begin{itemize}
%		\item \texttt{tsconfig.json} \textemdash~a quando a utilização de \textbf{TypeScript};
%		\item \texttt{jsconfig.json} \textemdash~a quando a utilização de \textbf{JavaScript}.
%	\end{itemize}
\end{mybox}

% ==> How to run React project <== %
\subsection{Execução do Projeto}

Após a criação do projeto é agora possível executar o mesmo, para tal é possível utilizar os \textit{\glslinkUnder{scripts}{scripts}} presentes no ficheiro \texttt{package.json}, sendo apenas necessário recorrer a um dos comandos que se segue (de acordo com o \textit{package manager} em uso):

\begin{itemize}
	\item \textbf{Yarn:} ~\texttt{yarn start};
	\item \textbf{NPM:} ~\texttt{npm start}
\end{itemize}

Se tudo correr como esperado será apresentado a seguinte mensagem no terminal:

\figureFrame{.5}{react-start.png}{Projeto \textbf{React} executado com sucesso}

\begin{mybox}{estg}{Nota}
	De notar que os comandos apresentados são para executar o projeto em modo de desenvolvimento, caso seja pretendido realizar o \textit{build} para colocar o projeto em produção os comandos a executar são:

	\begin{itemize}
		\item \textbf{Com Yarn:} ~\texttt{yarn build};
		\item \textbf{Com NPM:} ~\texttt{npm run build}.
	\end{itemize}
\end{mybox}