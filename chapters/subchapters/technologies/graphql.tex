\section{GraphQL}

\begin{minipage}[t]{.3\textwidth}
	\figureFrame{.5}{graphql.png}{\textbf{GraphQL} \textemdash~logo}
\end{minipage}
\begin{minipage}[t]{.7\textwidth}
	\minipagerestore
	\textbf{GraphQL} é uma \textit{query language open source} criada pelo \textbf{Facebook} tendo como principais objetivos tornar as \acrshortplUnder{api} mais rápidas, flexíveis e intuitivas. Além disso o \textbf{GraphQL} traz consigo um \glsShortUnder{ide}, chamado \textbf{\href{https://github.com/graphql/graphiql}{GraphiQL}}, que permite testar \textit{queries} e analisar o seu resultado no próprio \textit{browser}.

	A imagem apresentada abaixo demonstra a utilização do \textbf{GraphiQL}, onde do lado esquerdo são apresentadas as \textit{queries} e do lado direito o resultado das mesmas.
\end{minipage}

\figureFrame{1}{graphiql.jpg}{Demonstração do \textbf{GraphiQL}}

Uma das principais vantagens do \textbf{GraphQL} em comparação a um arquitetura \glsShortUnder{rest} é a capacidade de apenas requisitarem os dados que precisam num único pedido. Além disso, o \textbf{GraphQL} pode ser utilizado tanto em \glslinkUnder{backend}{\textit{back-end}} como em \glslinkUnder{frontend}{\textit{front-end}}, recorrendo para tal ao \href{https://www.apollographql.com/docs/apollo-server/}{\textbf{Apollo Server}} para \glslinkUnder{backend}{\textit{back-end}} e ao \textbf{\href{https://www.apollographql.com/docs/react/}{Apollo Client}} para \glslinkUnder{frontend}{\textit{front-end}}.

O excerto de código abaixo apresenta um exemplo de \textit{query} retirada da \href{https://graphql.org/learn/queries/}{documentação oficial},onde é possível analisar, de uma forma muito abstrata, que é pedido o nome do herói, bem como o nome dos seus amigos (\texttt{friends}).

\begin{longlisting}
	\inputminted{text}{code/graphql/example-query.graphql}
	\caption{\textbf{GraphQL} \textemdash~Exemplo de \textit{query}}
\end{longlisting}

Por sua vez, o excerto de código que se segue apresenta o resultado desta \textit{query}, sendo este apresentado no formato de um objeto \textbf{JSON}, contento a propriedade \texttt{data}, possuindo todos os resultados obtidos.

\begin{longlisting}
	\inputminted{json}{code/graphql/example-result.json}
	\caption{\textbf{GraphQL} \textemdash~Exemplo de resposta à \textit{query} realizada}
\end{longlisting}

\subsection{Instalação}

A instalação do \textbf{GraphQL} pode ser realizada através do \textbf{NPM} ou do \textbf{Yarn}, para isso basta recorrer a um dos seguintes comandos:

\begin{itemize}
	\item \textbf{Com Yarn:} ~\texttt{yarn add graphql}
	\item \textbf{Com NPM:} ~\texttt{npm install graphql}
\end{itemize}

Desta forma o \textbf{GraphQL} está disponível para utilizar ao longo do projeto recorrendo a uma das seguintes formas apresentadas abaixo.

\begin{longlisting}
	\begin{minted}{js}
		const { graphql, buildSchema } = require('graphql');
	\end{minted}

	\caption{Importação do \textbf{GraphQL} em \textbf{JavaScript}}
\end{longlisting}

\begin{longlisting}
	\begin{minted}{js}
		import { graphql, buildSchema } from 'graphql';
	\end{minted}

	\caption{Importação do \textbf{GraphQL} em \textbf{TypeScript}}
\end{longlisting}