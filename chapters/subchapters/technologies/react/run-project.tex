\subsection{Execução do Projeto}

Após a criação do projeto é agora possível executar o mesmo, para tal é possível utilizar os \textit{\glslinkUnder{script}{scripts}} presentes no ficheiro \texttt{package.json}, sendo apenas necessário recorrer a um dos comandos que se segue (de acordo com o \textit{package manager} em uso):

\begin{itemize}
	\item \textbf{Yarn:} ~\texttt{yarn start};
	\item \textbf{NPM:} ~\texttt{npm start}
\end{itemize}

Se tudo correr como esperado será apresentado a seguinte mensagem no terminal:

\figureFrame{.5}{react-start.png}{Projeto \textbf{React} executado com sucesso}

\begin{mybox}{estg}{Nota}
	De notar que os comandos apresentados são para executar o projeto em modo de desenvolvimento, caso seja pretendido realizar o \textit{build} para colocar o projeto em produção os comandos a executar são:

	\begin{itemize}
		\item \textbf{Com Yarn:} ~\texttt{yarn build};
		\item \textbf{Com NPM:} ~\texttt{npm run build}.
	\end{itemize}
\end{mybox}