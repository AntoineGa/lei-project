\subsection{Variáveis}

No \textbf{\glsShortUnder{sass}} é possível declarar variáveis recorrendo ao símbolo \verb|$| seguido do nome pretendido. Nos excertos de código que se seguem é possível analisar a declaração de variáveis, o seu uso e qual o resultado após este ser compilado para \textbf{\glsShortUnder{css}}.

\begin{longlisting}
	\begin{minted}[highlightlines={5,6},highlightcolor=yellow!25]{sass}
		$padding: 10px 20px
		$defaultColor: #ca4d24

		.alert
			background-color: $defaultColor
			padding: $padding
	\end{minted}
	\caption{Utilização de variáveis em \textbf{Sass}}
\end{longlisting}

No \textbf{\glsShortUnder{css}} estas variáveis não são visíveis, uma vez que o valor destas serão apresentadas diretamente na linha da sua utilização, ou seja:

\begin{longlisting}
	\begin{minted}[highlightlines={2,3},highlightcolor=yellow!25]{css}
		.alert {
			background-color: #ca4d24;
			padding: 10px 20px;
		}
	\end{minted}
	\caption{Código \textbf{CSS} resultante do excerto de código com variáveis em \textbf{Sass}}
\end{longlisting}

Porém em \textbf{\glsShortUnder{css}} também é possível utilizar variáveis, porém estas são definidas recorrendo a \verb|:root {}|. No excerto de código que se segue é apresentado um exemplo de variáveis em \textbf{\glsShortUnder{css}}.

\begin{longlisting}
	\begin{minted}{css}
		:root {
			--padding: 10px 20px;
			--default-color: #ca4d24
		}

		.alert {
			padding: var(--padding);
			background-color: var(--default-color);
		}
	\end{minted}
	\caption{Declaração e uso de variáveis em \textbf{CSS}}
\end{longlisting}