\section{Ambiente de Desenvolvimento}

% ==> IDE <== %
\subsection{IDE}

O \glsShortUnder{ide} é a ferramenta com mais destaque no processo de desenvolvimento, visto ser através deste que será escrito todo o código.

No caso do \glsShortUnder{ide} não existe nenhuma obrigatoriedade sobre qual usar, o programador deve escolher qual o \glsShortUnder{ide} com que se identifica mais, conseguindo assim optimizar o seu \textit{workflow}.

\subsection{Visual Studio Code}

\begin{minipage}{.3\textwidth}
	\figureFrame{.45}{vscode.png}{\textbf{Visual Studio Code} \textemdash~logo}
\end{minipage}
\begin{minipage}{.7\textwidth}
	\minipagerestore
	O \textbf{Visual Studio Code} é por norma o \glsShortUnder{ide} de preferência de muitos programadores e, isso deve-se essencialmente à sua versatilidade e às diversas extensões disponíveis para o mesmo.

	Em \hyperrefUnder{vscodeAttachments}{anexo} é possível encontrar a configuração utilizada no \textbf{Visual Studio Code} durante a realização deste projeto. Além destas configurações, é ainda possível encontrar as seguintes referências sobre a configuração e uso deste \glsShortUnder{ide} para desenvolvimento \textbf{JavaScript} e \textbf{React}: \cite{ultimateVSReact,reactToolsVS,spVSExtensions,vscodeReactSP}
\end{minipage}
\subsubsection{WebStorm}

\begin{minipage}{.3\textwidth}
	\figureFrame{.45}{webstorm.png}{\textbf{WebStorm} \textemdash~logo}
\end{minipage}
\begin{minipage}{.7\textwidth}
	\minipagerestore
	O \textbf{WebStorm} é outro \glsShortUnder{ide} bastante conhecido e ``poderoso'', não sendo v		necessário instalar \textit{plugins}/extensões devido a este ser bastante completo.

	Este \glsShortUnder{ide} faz parte das muitas ferramentas disponibilizadas pela \textbf{JetBrains}, tendo como principal vantagem a capacidade de \textit{\glslinkUnder{autocomplete}{autocomplete}} sem a necessidade de \textit{plugins}/extensões adicionais.

\end{minipage}



% ==> Package Manager <== %
\subsection{Gestor de Pacotes}

Como gestor de pacotes, ou \textit{package manager}, podem ser utilizadas duas soluções, sendo elas o \textbf{\glsShortUnder{npm}} e o \textbf{Yarn}. Ambos possuem o mesmo objetivo, a gestão de pacotes num projeto, sendo que o \textbf{\glsShortUnder{npm}} vem incluso na instalação no \textbf{NodeJS}, já por sua vez o \textbf{Yarn} necessita de ser instalado posteriormente.

O \textbf{Yarn} conta com algumas melhorias em relação ao \textbf{\glsShortUnder{npm}}, na tabela que se segue é possível analisar uma pequena comparação entre ambos.

\begin{table}[h!]
	\renewcommand{\arraystretch}{1.25}
	\centering
	\begin{tabularx}{.85\textwidth}{ |c X X X| }
		\rowcolor{estg} & {\color[HTML]{FFFFFF} \textbf{Sem Cache}} & 	{\color[HTML]{FFFFFF} \textbf{Com Cache}} & {\color[HTML]{FFFFFF} \textbf{Reinstalar}} \\\hline


		\textbf{NPM 6.13.4} & 67 segudos & 61 segundos & 28 segundos \\\hline
		\textbf{Yarn 1.21.1} & 57 segundos & 29 segundos & 1.2 segundos \\\hline
	\end{tabularx}

	\caption{Principais diferenças entre \textbf{TypeScript} e \textbf{JavaScript}}
\end{table}