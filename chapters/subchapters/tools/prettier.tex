\subsection{\textbf{Prettier} \textemdash~formatação de código}

\begin{minipage}{.3\textwidth}
	\figureFrame{.45}{prettier.png}{\textbf{Prettier} \textemdash~logo}
\end{minipage}
\begin{minipage}{.7\textwidth}
	\minipagerestore
	O \textbf{Prettier} é um \glslinkUnder{packages}{package} destinado à formatação do código auxiliando os desenvolvedores durante todo o processo de desenvolvimento, permitindo criar um ficheiro de configuração com todas as regras que serão aplicadas a quando a formatação do código.

	O \textbf{Prettier} suporta linguagens/\textit{frameworks} como \textbf{JavaScript}, \textbf{JSX}, \textbf{Markdown}, \textbf{HTML}, \textbf{CSS}, \textbf{Less}, entre outros.
\end{minipage}

Juntamente com o \textbf{Prettier} pode ainda ser utilizando os seguintes \glslinkUnder{packages}{packages}:

\begin{itemize}
	\item \textbf{\href{https://github.com/typicode/husky}{Husky}:} permitindo a definição de \textit{hooks} a realizar na execução de comandos do \textbf{Git};
	\item \textbf{\href{https://github.com/okonet/lint-staged}{Lint Staged}:} para executar \textit{hooks} apenas em ficheiros modificados com determinadas extensões (por exemplo \verb|.tsx|);
\end{itemize}

Em \hyperrefUnder{prettierAttachments}{anexo} é possível encontrar mais detalhes sobre o uso de \textbf{Prettier} juntamente com o \textbf{Husky}, apresentado ainda como realizar algumas configurações no ficheiro \verb|.prettierrc|.