\section{Trabalhos Futuros}

Ao longo do projeto foram surgindo formas de melhorar o projeto, bem como pontos que seriam importantes abordar, porém o tempo era limitado, não sendo possível avaliar o esforço que implicaria realizar algumas destas e posteriormente implementar as mesmas. Nos paragráfos seguintes é possível encontrar alguns dos trabalhos que poderiam ser realizados futuramente de forma a melhorar o projeto na sua totalidade.

\subsection{Biblioteca de Componentes}

Um dos pontos seria a criação de uma biblioteca de componentes, isto deve-se essencialmente a existerem componentes iguais em ambas as vertentes (\textit{backoffice} e \textit{frontoffice}), evitando também assim o termo \textbf{\textit{\glsShortUnder{dry}}}, ou seja, usar ``pedaços'' de código iguais, mas em vários locais.

A criação desta biblioteca facilitaria o uso de componentes comuns a ambas as vertentes do projeto (\textit{backoffice} e \textit{frontoffice}), sendo esta instalada através do \textbf{\glsShortUnder{npm}} ou \textbf{Yarn}. Para realizar esta biblioteca existem várias ferramentas, podendo ser criando um projeto \textbf{React}, em que posteriormente este seria publicado como \glslinkUnder{packages}{package}.

\textbf{Referências:} \cite{publishReactPackage,createLibReact}

\subsection{Testes}

Os testes seriam outra das melhorias a implementar, auxiliando na validação do projeto.  Seriam possível realizar dois tipos de testes, os \textit{unit tests} ou testes unitários e ainda testes \textit{end-to-end}.

Nos tópicos que se seguem é possível analisar melhor cada um destes tipos de testes.

\vspace{10pt}

\textbf{Referênciais Adicionais:} \cite{reactEndToEndGuide,jestReact,endToEndCypress,endToEndJestPuppeteer,endToEndWJest,unitTestsReact,reactTesting,modernCypressTesting}

\subsubsection{\textit{Unit Testing}}

Os testes unitários, ou \textit{unit tests}, são testes focados em partes isoladas de um projeto ou sistema, por exemplo um componente \textbf{React}. O principal objetivo deste tipo de testes é auxiliar na validação do código produzido e a encontrar \textit{bugs} que não tenham surgido até então.

Uma das ferramentas frequentemente utilizada para \textit{unit tests} é o \textbf{\href{https://jestjs.io/}{Jest}}, podendo recorrer ou não a outros \textit{\glslinkUnder{packages}{packages}} adicionais. O \textbf{Jest} possibilita realizar testas em diversas \textit{frameworks} e tecnologias, contando ainda com \textit{code coverage}, ou seja, cobertura de código, sendo apresentado no formato de tabela como na figura\footnote{Retirada do \href{https://jestjs.io/}{site oficial}} que se segue.

\figureFrame{.75}{jest-code-coverage.jpeg}{\textbf{Jest} \textemdash~cobertura de código}

Um teste em \textbf{Jest} poderia ser escrito da seguinte forma:

\begin{longlisting}
	\begin{minted}[]{jsx}
		import ReactDOM from 'react-dom'
		import Component from 'components/Component';

		it('renders without crashing', () => {
			const div = document.createElement('div');
			ReactDOM.render(<Component />, div);
			ReactDOM.unmountComponentAtNode(div);
		});
	\end{minted}

	\caption{\textbf{Jest} \textemdash~exemplo de um teste para um componente \textbf{React}}
\end{longlisting}

\begin{mybox}{estg}{Nota}
	Para executar os testes e validar tanto o teste como o código produzido seria necessário executar o comando:

	\begin{itemize}
		\item \textbf{Com NPM:} ~\verb|npm test|;
		\item \textbf{Com Yarn:} ~\verb|yarn test|.
	\end{itemize}

	É importante ainda referir que em determinados casos pode ser necessário criar o \textit{script} no ficheiro \texttt{package.json} para executar o \textbf{Jest} e os testes criados.
\end{mybox}

\subsubsection{\textit{End-to-end Testing}}

Os testes \textit{End-to-End} são o contrário dos testes unitários, ou seja, validam o projeto como um todo, sendo utilizados frequentemente para validar o fluxo seguido pelo projeto. Uma ferramente que pode ser utilizada para realizar este tipo de testes é o \textbf{\href{https://cypress.io}{Cypress}}.

O excerto de código que se segue é um exemplo de teste\footnote{Retirado da \href{https://www.cypress.io/blog/2019/02/05/modern-frontend-testing-with-cypress/}{blog oficial}} realizado com \textbf{Cypress}, sendo possível analisar logo de seguida o seu resultado.

\begin{longlisting}
	\begin{minted}{js}
		it('should add a new todo to the list', () => {
			// network stubs
			cy.server();
			cy.route('GET', `${serverUrl}/todos`, '@updatedJSON').as('getAllTodos');
			cy.route('POST', `${serverUrl}/todos`, '@addTodoJSON').as('addTodo');
		});
	\end{minted}
	\caption{\textbf{Cypress} \textemdash~exemplo de teste}
\end{longlisting}

\figureFrame{.75}{cypress-test.jpeg}{\textbf{Cypress} \textemdash~resultado do teste apresentado}

Também é possível executar testes \textit{End-to-End} com \textbf{Jest} e \textbf{Puppeeteer}\footnote{\textbf{Referências:} \cite{endToEndJestPuppeteer,endToEndWJest}}, sendo uma alternativa ao \textbf{Cypress}.

