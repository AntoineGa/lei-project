\tocChap{Convenções e Nomenclatura}

Ao longo deste relatório, optou-se por seguir um conjunto de convenções de forma a facilitar a interpretação do texto, exemplos e excertos de código apresentados.

Desta forma textos em \textit{itálico} terão como objetivo representar estrangeirismos, já textos em \textbf{negrito} terão como objetivo realçar termos com maior relevância ou mesmo nomes de empresas, marcas, etc..

Já em casos de textos \underline{sublinhados}, por norma, referem-se a ligações no documento, por exemplo a ligação para uma determinada definição no glossário.

Além disso, sempre que seja pretendido realçar uma nota será utilizado o exemplo abaixo.

Contudo e, sempre que seja pertinente realçar uma determinada nota, será utilizado o formado que é apresentado de seguida.

\vspace{0.01cm}

\begin{mybox}{estg}{Nota}
	Informação da nota
\end{mybox}

\vspace{0.1cm}

Porém e, recorrendo ao esquema anterior, sempre que seja necessário apresentar informações sobre um erro que poderá ocorrer ou que ocorreu, será utilizado o formato apresentado abaixo.

\vspace{0.01cm}

\begin{errorbox}{Erro Apresentado}
Mensagem ou informações sobre o erro.
\end{errorbox}

\vspace{0.1cm}

Sempre que seja pertinente adicionar determinada citação, será utilizado o formato apresentado abaixo.

\begin{flushright}
	\begin{quotebox50}
		``Citação''

		\tcblower

		Autor ou Referência da citação
	\end{quotebox50}
\end{flushright}

\vspace{0.1cm}

No caso de excertos de código e, de forma a manter a \textit{syntax} o mais correta possível, será utilizado o formato apresentado abaixo, sendo possível visualizar os números das linhas, bem como, caso seja pertinente, destacar alguma destas linhas.

\vspace{0.01cm}

\begin{longlisting}
	\begin{minted}{js}
		// Exemplo de Excerto de Código
		console.log("Hello World");
	\end{minted}
	\caption{Demonstração de excerto de código}
\end{longlisting}

\vspace{0.1cm}

No que toca a nomenclatura e, tal como será possível analisar ao longo deste documento, são seguidas as seguintes regras:

\begin{itemize}
	\item \textbf{Componentes React:} nomes em \textit{\textbf{Pascal Case}}, ou seja, a primeira letra do identificador e a primeira letra de cada palavra são escritas em maiúsculas;
	\item \textbf{Interfaces:} seguem novamente o \textit{naming convention \textbf{Pascal Case}} e começam pela letra \textbf{I}, que representa interface;
\end{itemize}