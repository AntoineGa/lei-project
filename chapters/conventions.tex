\chapter*{Conveções \& Nomenclatura}
\addcontentsline{toc}{chapter}{Conveções \& Nomenclatura}  

Ao longo deste relatório, optou-se por seguir um conjunto de convenções de forma a facilitar a interpretação do texto, exemplos e exercertos de código apresentados.

Desta forma textos em \textit{itálico} terão como objetivo representar estrangeirismos, já textos em \textbf{negrito} terão como objetivo realçar termos com maior relevância ou mesmo nomes de empresas, marcas, etc..

Além disso, sempre que seja pretendido realçar uma nota será utilizado o exemplo abaixo.\\[0.01cm]

\begin{mybox}{estg}{Nota}
	Informação da nota
\end{mybox}

\vspace{0.1cm}

Já para apresentar excertos de código ao longo deste relatório, optou-se por utilizar o seguinte esquema:\\[0.01cm]

\begin{lstlisting}[language=Javascript]
// Excerto de Código de Excemplo
console.log("Hello World");
\end{lstlisting}

\vspace{0.1cm}

No que toca a nomenclatura e, tal como será possível analisar ao longo deste documento, são seguidas as seguintes regras:

\begin{itemize}
	\item \textbf{Componentes React:} nomes em \textit{\textbf{Pascal Case}}, ou seja, a primeira letra do identificador e a primeira letra de cada palavra são escritas em maiúsculas;
	\item \textbf{Interfaces:} seguem novamente o \textit{naming convention \textbf{Pascal Case}} e começam pela letra \textbf{I} que representa interface;
\end{itemize}