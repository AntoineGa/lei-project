\chapter{Ambiente de Desenvolvimento}

% ==> IDE <== %
\section{IDE}

O \glsShortUnder{ide} é a ferramenta com mais destaque no processo de desenvolvimento, visto ser através deste que será escrito todo o código.

No caso do \glsShortUnder{ide} não existe nenhuma obrigatoriedade sobre qual usar, o programador deve escolher qual o \glsShortUnder{ide} com que se identifica mais, conseguindo assim optimizar o seu \textit{workflow}.

\subsection{Visual Studio Code}

\begin{minipage}{.3\textwidth}
	\figureFrame{.45}{vscode.png}{\textbf{Visual Studio Code} \textemdash~logo}
\end{minipage}
\begin{minipage}{.7\textwidth}
	\minipagerestore
	O \textbf{Visual Studio Code} é por norma o \glsShortUnder{ide} de preferência de muitos programadores e, isso deve-se essencialmente à sua versatilidade e às diversas extensões disponíveis para o mesmo.

	Em \hyperrefUnder{vscodeAttachments}{anexo} é possível encontrar a configuração utilizada no \textbf{Visual Studio Code} durante a realização deste projeto. Além destas configurações, é ainda possível encontrar as seguintes referências sobre a configuração e uso deste \glsShortUnder{ide} para desenvolvimento \textbf{JavaScript} e \textbf{React}: \cite{ultimateVSReact,reactToolsVS,spVSExtensions,vscodeReactSP}
\end{minipage}
\subsubsection{WebStorm}

\begin{minipage}{.3\textwidth}
	\figureFrame{.45}{webstorm.png}{\textbf{WebStorm} \textemdash~logo}
\end{minipage}
\begin{minipage}{.7\textwidth}
	\minipagerestore
	O \textbf{WebStorm} é outro \glsShortUnder{ide} bastante conhecido e ``poderoso'', não sendo v		necessário instalar \textit{plugins}/extensões devido a este ser bastante completo.

	Este \glsShortUnder{ide} faz parte das muitas ferramentas disponibilizadas pela \textbf{JetBrains}, tendo como principal vantagem a capacidade de \textit{\glslinkUnder{autocomplete}{autocomplete}} sem a necessidade de \textit{plugins}/extensões adicionais.

\end{minipage}



% ==> Package Manager <== %

% Yarn

% NPM

% ==> Git & Version Control <== %
\section{Controlo de Versões}

Durante o desenvolvimento de todo o projeto foi utilizado o \textbf{GitLab} para controlo de versões, usufruindo de todas as funcionalidades que este oferece. Nos pontos que se seguem será possível analisar toda a parte relativa ao \textbf{GitLab}, desde da criação e atribuição de \textit{issues}, bem como a sua resolução implicando para tal a criação de uma nova \textit{branch} e de um \textit{merge request}.

\subsection{\textit{Board}}

A \textit{board} do \textbf{GitLab} é bastante versátil, permitindo criar \textit{labels} de forma a organizar todas as tarefas, bem como indicar o seu estado atual. Na imagem que se segue é possível analisar a \textit{board} existente para este projeto, bem como as respetivas \textit{labels}.

Na \textit{board} utilizada no decorrer do projeto é possível encontrar as seguintes colunas:

\begin{itemize}
	\item \textbf{\textit{Open}} \textemdash~coluna por defeito do \textbf{GitLab} para todas as \textit{issues} pendentes;
	\item \textbf{\textit{Closed}} \textemdash~coluna por defeito do \textbf{GitLab} para todas as \textit{issues} encerradas/terminadas;
	\item \textbf{\textit{To Do}} \textemdash~coluna com todas as \textit{issues} pendentes;
	\item \textbf{\textit{In Progress}} \textemdash~coluna com todas as \textit{issues} em progresso;
	\item \textbf{\textit{Review}} \textemdash~coluna com todas as \textit{issues} que aguardam análise e futura aprovação.
\end{itemize}

A imagem que se segue apresenta a \textit{board} utilizada no decorrer do projeto.

% TODO add board image

\subsection{\textit{Issues}}

\subsection{\textit{Merge Requests}}