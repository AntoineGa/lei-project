\tocChap{Resumo}

No âmbito da unidade curricular \textbf{Projeto Final} e, ao longo dos últimos três meses, foi desenvolvida uma aplicação \textit{web} destinada a treinadores e atletas de alta competição. Este projeto já se encontra em funcionamento, porém surgiu a necessidade da criação de novas funcionalidades, bem como um \textit{layout} mais apelativo e moderno.

A principal necessidade do projeto é proporcionar tanto aos treinadores, como atletas, uma melhor experiência de utilização e interação com o projeto, facilitando assim a consulta e gestão de treinos, bem como o acompanhamento. Assim e, através de gráficos e \textit{dashboards}, atletas e treinadores conseguem facilmente visualizar a evolução de um atleta, bem como as suas demais informações.

O projeto é constituído por uma arquitetura cliente-servidor, \textit{back-end} e \textit{front-end}, onde do lado do \textit{back-end} é possível encontrar tecnologias e ferramentas como \textbf{KoaJS}, \textbf{PostgreSQL}, \textbf{TypeORM}, \textbf{Apollo Server}, \textbf{GraphQL}, entre outras. Já do lado do \textit{front-end} é possível encontrar tecnologias e ferramentas como \textbf{React} e \textbf{Apollo Client}.

Em ambas as camadas desta arquitetura é possível encontrar o uso de \textbf{TypeScript}, existindo determinadas vantagens no seu uso como será possível analisar ao longo deste documento.

A camada do cliente, ou seja, o \textit{front-end}, é separa em duas vertentes, a vertente de \textit{backoffice}, sendo esta destinada aos treinadores e administrador(es) e, a vertente de \textit{frontoffice} destinada aos atletas. Ao longo do documento será possível encontrar em mais detalhe as funcionalidades presentes em cada vertente, bem como o \textit{layout} de cada.

No final e, apesar do projeto não se encontrar concluído, existe, páginas de cada vertente funcionais e prontas a comunicar com o \textit{back-end} do projeto e realizar todas as ações necessárias.

Por fim referir, que apesar do projeto contar com a camada do lado do servidor, o \textit{back-end}, o foco do projeto é a camada do lado do cliente, o \textit{front-end}.

\vfill

\textbf{Palavras-chave:} \textit{React}, Desenvolvimento \textit{Web}, \textit{Front-end}, BeAPT

\newpage