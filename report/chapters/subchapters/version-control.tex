\section{Controlo de Versões}

Para controlo de versões e alterações foi utilizado o \textbf{GIT} em conjuto com o \textbf{GitLab}, sendo seguido o \textit{workflow} apresentado na imagem que se segue.

\figureFrame{1}{git-workflow.png}{Workflow seguido no \textbf{GitLab} durante o desenvolvimento}

Como é possível analisar, existe uma \textit{branch} principal, normalmente com o nome \textbf{Master} ou \textbf{Main} que contém essencialmente versões do projeto. Durante o desenvolvimento existe uma outra \textit{branch} destinada apenas ao desenvolvimento, que por sua vez são criadas \textit{branchs} através desta para a resolução de issues.

Ao criar uma \textit{branch} para a resolução de uma \textit{issue}, é criado também o \textit{merge request} para a mesma, ficando este em estao de \textit{\glsShortUnder{wip}} ou \textit{Draft} de forma a indicar que ainda existe trabalho em progresso.

Assim que a \textit{issue} é concluída, é removido do \textit{merge request} o estado de \textit{\glsShortUnder{wip}} ou \textit{Draft} sendo assim analisado e posteriormente realizado o \textit{merge} para a branch em questão.