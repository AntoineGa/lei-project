\subsection{NodeJS}

\begin{minipage}{.3\textwidth}
	\figureFrame{.5}{nodejs.png}{\textbf{NodeJS} \textemdash~logo}
\end{minipage}
\begin{minipage}{.7\textwidth}
	\minipagerestore
	\textbf{NodeJS} é um ambiente de execução \textbf{JavaScript}, \textit{open source}, que permite desenvolver aplicações do lado do servidor (\glslinkUnder{backend}{\textit{back-end}}). Desta forma é possível criar aplicações utilizando \textbf{JavaScript} que não necessitam de um \textit{browser} para a sua execução.

	A \textit{performance} do \textbf{NodeJS} deve-se essencialmente ao uso do interpretador \href{https://v8.dev}{\textbf{V8} da \textbf{Google}}, interpretador este que é o \textit{core} do \textbf{Google Chrome}.

\end{minipage}

Em \hyperrefUnder{nodeArch}{anexo {\footnotesize(página 82)}} é possível encontrar a imagem que representa a arquitetura do \textbf{NodeJS} em comparação com a arquitetura tradicional\footnote{\textbf{Imagem retirada de} \cite{caseByCaseNode}}.

Desta forma é possível analisar que no caso da arquitetura tradicional é criada uma nova \textit{thread} para cada pedido, já no \textbf{NodeJS}, apenas existe uma \textit{thread} que possui I/O não bloqueante, permitindo várias requisições simultâneas, ficando retidas no \textit{event-loop}.

Em 2020, segundo dados do \href{https://insights.stackoverflow.com/survey/2020#technology-most-loved-dreaded-and-wanted-other-frameworks-libraries-and-tools-loved3}{\textbf{StackOverflow}}, o \textbf{NodeJS} ficou em sétimo lugar na lista de outras \textit{frameworks}, bibliotecas ou ferramentas preferidas dos programadores e, em primeiro lugar na lista de outras \textit{frameworks}, bibliotecas ou ferramentas mais procuradas.

Além disso, é ainda possível encontrar em \hyperrefUnder{nodeAttachments}{anexo {\footnotesize(página 81)}} como realizar a instalação do \textbf{NodeJS} nos diversos sistemas operativos.