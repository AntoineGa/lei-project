\subsection{Gestor de Pacotes}

Como gestor de pacotes, ou \textit{package manager}, podem ser utilizadas duas soluções, sendo elas o \textbf{\glsShortUnder{npm}} e o \textbf{Yarn}. Ambos possuem o mesmo objetivo, a gestão de pacotes num projeto, sendo que o \textbf{\glsShortUnder{npm}} vem incluso na instalação no \textbf{NodeJS}, já por sua vez o \textbf{Yarn} necessita de ser instalado posteriormente.

O \textbf{Yarn} conta com algumas melhorias em relação ao \textbf{\glsShortUnder{npm}}, na tabela que se segue é possível analisar uma pequena comparação entre ambos\footnote{Retirado de \cite{yarnVSNpm}}.

\begin{table}[h!]
	\renewcommand{\arraystretch}{1.25}
	\centering
	\begin{tabularx}{.85\textwidth}{ |c X X X| }
		\rowcolor{estg} & {\color[HTML]{FFFFFF} \textbf{Sem Cache}} & 	{\color[HTML]{FFFFFF} \textbf{Com Cache}} & {\color[HTML]{FFFFFF} \textbf{Reinstalar}} \\\hline


		\textbf{NPM 6.13.4} & 67 segudos & 61 segundos & 28 segundos \\\hline
		\textbf{Yarn 1.21.1} & 57 segundos & 29 segundos & 1.2 segundos \\\hline
	\end{tabularx}

	\caption{Comparação entre \textbf{Yarn} e \textbf{NPM}}
\end{table}

Além das diferenças apresentadas acima, o \textbf{Yarn} conta com outras melhorias em comparação ao \textbf{\glsShortUnder{npm}}, como por exemplo:

\begin{itemize}
	\item Interface mais \textit{clean};
	\item Facilidade de uso \textemdash ~determinados comandos tornam-se mais intuitivos com o \textbf{Yarn};
	\item Possibilidade de reinstalar \glslinkUnder{packages}{packages} sem conexão à Internet.
\end{itemize}


Em \hyperrefUnder{yarnAttachments}{anexo {\footnotesize (página 79 e 80)}} é possível encontrar como proceder à instalação do \textbf{Yarn}.