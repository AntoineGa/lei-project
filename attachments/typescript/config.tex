\tocSubSec{Configuração}

O \textbf{TypeScript} permite realizar determinadas configurações no projeto, recorrendo para tal ao ficheiro \texttt{tsconfig.json}\footnote{\textbf{\href{https://www.typescriptlang.org/tsconfig}{Documentação Oficial}}}. Neste ficheiro e, tal como é possível visualizar no excerto de código abaixo, é possível definir configurações relacionadas com a estrutura de pastas, qual a versão do \glsShortUnder{es} a usar, entre outras configurações.

Além das configurações referidas anteriormente, entre muitas outras, é possível realizar a configuração de caminhos (\textit{paths}), permitindo assim manter todas as importações realizadas durante o projeto mais ``enxutas''. No excerto de código que se segue é possível analisar que foi criado um \textit{path} para a pasta \texttt{components}, desta forma sempre que seja realizada a importação de um componente é possível utilizar o \textit{path} \texttt{@components/} seguido do nome do componente.

\begin{longlisting}
	\inputminted{json}{code/typescript/tsconfig.json}
	\caption{\textbf{TypeScript} \textemdash~Ficheiro \texttt{tsconfig.json}}
\end{longlisting}

\begin{mybox}{estg}{Nota}
	O ficheiro \texttt{tsconfig.json} pode-se gerado automaticamente através dos comandos ~\texttt{npx tsc --init} ou então ~\texttt{yarn tsc --init}, sendo que este ficheiro gerado apenas trará todas as configurações possíveis, sendo necessário proceder posteriormente à sua correta configuração de acordo com o projeto em questão.

	\vspace{0.35cm}

	O ficheiro \texttt{tsconfig.json} apresentado tem como objetivo apresentar apenas uma possível estrutura de configuração. É recomendado consultar a \href{https://www.typescriptlang.org/tsconfig}{documentação oficial} relativa a este ficheiro.

	\vspace{0.15cm}

	É importante referir que este ficheiro deve encontra-se na raíz do projeto para garantir o seu correto funcionamento.
\end{mybox}