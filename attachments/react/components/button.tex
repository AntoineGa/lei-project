\tocSubSubSec{\textit{Button}}

O componente \textit{Button} é outros dos componentes presente em diversos ecrãs da aplicação, porém com diferentes estilos. Assim este componente recebe,tal como no componente anterior, proprieades que permitem tornar este componente mais versátil.

\begin{longlisting}
	\begin{minted}[]{js}
	const Button: React.FC<Props> = ({
		icon,
		className,
		rounded,
		to,
		type,
		modifier,
		children,
		onClick,
		...attributes
	}) => {
		// ..
	}
	\end{minted}
	\caption{Propriedades recebidas no componente \textbf{Button}}
\end{longlisting}

Neste ponto serão apenas apresentadas as proprieades \texttt{rounded} e \texttt{modifier}, visto serem as que causa maior impacto na aparência do componente. De referir que caso este receba a propriedade \texttt{to}, este utiliza um \verb|<Link>| em vez de \verb|<button>|, para isso:

\begin{longlisting}
	\begin{minted}[]{js}
		const Tag: ElementType = to ? Link : 'button';
		const tagAttributes = to ? { to } : { type };
	\end{minted}
	\caption{Uso de \texttt{Link} ou \texttt{button} no componente \textbf{Button}}
\end{longlisting}

De referir que para a propriedade \texttt{modifier} (que pode receber os valores: \verb|'alt-primary', 'info', 'danger'|) ou da propriedade \texttt{rounded} ter impacto, é definida uma \textit{class} \textbf{\glsShortUnder{css}} adicional, no excerto de código que segue é possível analisar a aplicação dessa(s) \textit{class(es)}.

\begin{longlisting}
	\begin{minted}[]{html}
		<Tag
			{...tagAttributes}
			{...attributes}
			onClick={onClick}
			className={`
				${styles['root'] || ''}
				${styles[`${modifier}-modifier`] || ''}
				${(rounded && styles['rounded']) || ''}
				${className || ''}`}
		></Tag>
	\end{minted}
	\caption{Aplicação de classes adicionais para as proprieades \texttt{rounded} e \textit{modifier}}
\end{longlisting}

Por sua vez, no estilo do componente é possível encontrar o seguinte estilo:

\begin{longlisting}
	\begin{minted}[]{sass}
	.alt-primary
		background-color: $whiteColor
		color: $buttonHighlightColor

		&:disabled
			color: $whiteColor
			background-color: $secondaryColor
			box-shadow: 0px 1px 2px #00000029

		&:enabled
			&:hover
				box-shadow: 0px 5px 6px #00000029

			&:focus
				background-color: $whiteColor
				color: $buttonHighlightColor
				box-shadow: 0px 0px 7px #13ACCC

			&:active
				background-color: $whiteColor
				color: $buttonHighlightColor
				border: 1px solid #D3D3D3
				box-shadow: none


	.info-modifier
		background: $appBackgroundColor
		color: $primaryColor

		&:disabled
			background-color: $secondaryColor
			color: $whiteColor

		&:enabled
			&:hover
				background-color: $appBackgroundColor
				color: #707070
				box-shadow: 0px 5px 6px #00000029

			&:focus
				background-color: $appBackgroundColor
				color: $primaryColor
				box-shadow: 0px 0px 7px #13ACCC

			&:active
				color: $primaryColor
				background-color: $appBackgroundColor
				box-shadow: none

	.danger-modifier
		background: $dangerColor
		color: $appForegroundColor

	.rounded
		border-radius: calc(#{$buttonHeight} / 2)
		padding: 0 1em
	\end{minted}

	\caption{Estilo aplicado para as propriedades \texttt{modifier} e \texttt{rounded}}
\end{longlisting}

Importante referir que no caso da propriedade ~\texttt{modifier}~ não ser aplicada, o componente \textbf{\textit{Button}} conta com um estilo de cores e sombras padrão.