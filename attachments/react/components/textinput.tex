\tocSubSubSec{\textit{TextInput}}

O Componente \textbf{\textit{TextInput}} é um componente destinado a ser utilizado nos diversos ecrãs do projeto. Por exemplo nos \textit{mockups} apresentados no \hyperrefUnder{loginSequenceDiagram}{diagrama de sequência de inicio de sessão}, onde este é utilizado para o campo de \textit{email} e \textit{password}.

Para isto ser possível o componente recebe as propriedades apresentadas no excerto de código que se segue.

\begin{longlisting}
	\begin{minted}{js}
	const TextInput = ({
		className,
		disabled,
		noValidate,
		onChange,
		type,
		name,
		placeholder,
		multiline,
		modifier,
		rounded,
		rows,
		...attributes
	}: ITextInput) => {
		// ...
	}
	\end{minted}

	\caption{Propriedades recebidas no componente \textbf{TextInput}}
\end{longlisting}

Assim sendo e, apesar de algumas das propriedades não serem obrigatórias, as que tem mais impacto no aspeto visual do \textit{input} são a propriedade \texttt{rounded}, \texttt{multiline} and \texttt{modifier}.

A propriedade \texttt{rounded} tem como objetivo definir se o \textit{input} conta com bordas redondas ou não. No caso desta propriedade ser aplicada, o \textit{input} recebe uma \textit{class} de \textbf{\glsShortUnder{css}} adicional (de forma a ser estilizada posteriormente), o excerto de código que segue apresenta a aplicabilidade desta propriedade, bem como o estilo adicionado.

\begin{longlisting}
	\begin{minted}[]{js}
	// ...
	<Tag
		className={`${styles[`${modifier}-style`] || ''} ${
			rounded && styles['rounded']
		}`}
		{...attributes}
		type={type}
		name={name}
		disabled={disabled}
		placeholder='&nbsp;'
		rows={rows ? Number(rows) : 0}
		onChange={onChange}
	>
	// ...
	</Tag>

	// ...
	\end{minted}

	\caption{Propriedade \texttt{rounded} aplicada ao input}
\end{longlisting}

\begin{longlisting}
	\begin{minted}[]{sass}
	.rounded
		border-radius: $inputBorderRadius // 25px
	\end{minted}

	\caption{Estilo adicionado na class da propriedade \texttt{rounded}}
\end{longlisting}

Já no caso da propriedade \texttt{multiline}, esta destina-se a dizer se será apresentado um \textit{input} ou \textit{textarea}, para isso é utilizado o código abaixo:

\begin{longlisting}
	\begin{minted}[highlightlines={3},highlightcolor=yellow!25]{js}
		// ...

		const Tag: ElementType = multiline ? 'textarea' : 'input';

		// ...
	\end{minted}

	\caption{Utilização da propriedade \texttt{multiline} no componente \textbf{TextInput}}
\end{longlisting}

Por fim, a propriedade \texttt{modifier} tem como objetivo afetar as cores que são aplicadas ao \textit{input}, no momento, para além do estilo padrão, a propriedade \texttt{modifier} pode receber o valor \textit{light}, aplicando assim o seguinte estilo ao \textit{input}.

\begin{longlisting}
	\begin{minted}[]{sass}
	&.light-style
		background-color: $whiteColor // #FFF

		&:hover
			border: 1px solid $secondaryColor // #b8b7b7

		&:focus
			outline: none
			border: 1px solid $secondaryColor  // #b8b7b7
	\end{minted}

	\caption{Estilo aplicado no uso da propriedade \texttt{modifier} com o valor light}
\end{longlisting}

Com estas propriedades o componente \textit{\textbf{TextInput}} torna-se bastante versátil, possibilitando o seu uso nos mais diversos formulários do projeto.