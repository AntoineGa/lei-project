\tocSubSec{Estrutura de Pastas}

A estrutura de pastas para um projeto \textbf{React} pode variar de projeto para projeto, ou da forma como o programador prefere organizar os mais diversos ficheiros do projeto. Porém e, tal como é possível analisar na figura que se segue, é comum encontrar a seguinte estrutura de pastas.

\begin{figure}[h!]
\centering
\begin{forest}
	for tree={
	  folder,
	  font=\ttfamily,
	  grow'=0
	}
	[{Raíz do Projeto}
	   [{public}]
	   [src
		  [assets]
		  [components]
		  [controllers]
		  [hooks]
	   ]
	]
\end{forest}
\caption{\textbf{React} \textemdash~possível estrutura de pastas}
\end{figure}

Importante referir que a pasta \texttt{src/} será a pasta principal, uma vez irá conter todos os componentes, \textit{assets} e outros ficheiros importantes para o projeto.

É importante referir que seguindo o método de criação do projeto \textbf{React} com o \texttt{create-react-app}, a estrutura de pastas e os ficheiros criados inicialmente será a seguinte:

\begin{figure}[h!]
\centering
\begin{forest}
	for tree={
		folder,
		font=\ttfamily,
		grow'=0
	}
	[{Raíz do Projeto}
		[public
			[index.html]
			[favicon.ico]
		]
		[src
			[App.css]
			[App.js]
			[App.test.js]
			[index.css]
			[index.js]
			[logo.svg]
		]
	]
\end{forest}
\caption{\textbf{React} \textemdash~estrutura de pastas e ficheiros gerados pelo \texttt{create-react-app}}
\end{figure}

\vspace{0.25cm}
\begin{mybox}{estg}{Nota}
	É importante relembrar que consoante o uso de \textbf{TypeScript} ou \textbf{JavaScript} será possível encontrar ficheiros \texttt{.tsx} ou \texttt{.ts}, \texttt{.jsx} ou \texttt{js}.

%	Além da extensão dos ficheiros será possível encontrar na raíz do projeto um ficheiro de configuração, podendo ser:
%	\begin{itemize}
%		\item \texttt{tsconfig.json} \textemdash~a quando a utilização de \textbf{TypeScript};
%		\item \texttt{jsconfig.json} \textemdash~a quando a utilização de \textbf{JavaScript}.
%	\end{itemize}
\end{mybox}