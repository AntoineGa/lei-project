\section*{\textbf{Visual Studio Code} \textemdash~Configurações}
\label{vscodeConfigs}
\addcontentsline{toc}{section}{\textbf{Visual Studio Code} \textemdash~Configurações}

\begin{longlisting}
	\inputminted{json}{code/vscode-settings.json}
	\caption{Configurações utilizadas no \textbf{Visual Studio Code}}
\end{longlisting}

\begin{mybox}{estg}{Nota}
	Para utilizar as Configurações apresentadas devem ser seguidos os passos abaixo:

	\begin{enumerate}
		\item Aceder às configurações do \textbf{Visual Studio Code} no formato \textbf{JSON}, para isso utilizar a tecla de atalho apresentada abaixo de acordo com o sistema operativo e pesquisar pela opção \textit{``Preferences: Open Settings (JSON)''};
		\begin{itemize}
			\item \textbf{No macOS:} \texttt{CMD + SHIFT + P};
			\item \textbf{No Windows/Linux:} \texttt{CTRL + SHIFT + P}.
		\end{itemize}
		\item Copiar as configurações apresentadas e colar no ficheiro \texttt{settings.json} (ficheiro que abriu no passo anterior).
		\begin{itemize}
			\item \textbf{Nota:} caso já possua configurações neste ficheiro, basta remover as chavetas inicias (\verb|{}|) no código apresentado e colocar as restantes configurações.
		\end{itemize}
	\end{enumerate}
\end{mybox}