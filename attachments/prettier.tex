\tocSec{\textbf{Prettier}}
\label{prettierAttachments}

\tocSubSec{Configuração}

\begin{longlisting}
	\inputminted{json}{code/prettier/config.json}
	\caption{Configurações utilizadas no \textbf{Prettier}}
\end{longlisting}

O código apresentado acima é do ficheiro \verb|.prettierrcc|, onde é possível definir várias configurações para o \textbf{Prettier}. Além destas configurações é possível ainda definir um ficheiro parecido com o \verb|.gitignore|, no caso \verb|.prettierignore|, onde tal como no \verb|.gitignore| são definidos os ficheiros ou pastas nas quais não será executado o \textbf{Prettier}.

\begin{longlisting}
	\begin{minted}{bash}
		package.json
		dist/
	\end{minted}
	\caption{Exemplo do conteúdo do ficheiro \texttt{.prettierignore}}
\end{longlisting}


\tocSubSec{\textbf{Husky} e \textbf{Git} \textit{hooks}}

Juntamente com o \textbf{Prettier} pode ser utilizado o \textbf{Husky} e o \textbf{Lint Staged} para formatar todo o código produzido ao realizar um \textit{commit}.

\begin{longlisting}
	\inputminted{json}{code/prettier/husky.json}
	\caption{Configurações utilizadas no \textbf{Husky}, \textbf{Lint Staged} e \textbf{Prettier}}
\end{longlisting}

Como é possível analisar, ao realizar um \textit{commit} será executado o \textbf{Lint Staged}, que por sua vez apenas aplica os comandos aos ficheiros que terminem com as extensões definidas.

Desta forma todo o código é formatado seguindo as regras definidas no ficheiro \verb|.prettier|.