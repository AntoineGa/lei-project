% ==> Acronyms <== %
\newacronym{lei}{LEI}{Licenciatura em Egenharia Informática}

\newacronym{estg}{ESTG}{Escola Superior de Tecnologia e Gestão}

\newacronym{json}{JSON}{\textit{JavaScript Object Notation}}

\newacronym{yaml}{YAML}{\textit{YAML Ain't Markup Language}}

\newacronym{sql}{SQL}{\textit{Structured Query Language}}

\newacronym{html}{HTML}{\textit{Hyper Text Markup Language}}

\newacronym{css}{CSS}{\textit{Cascading Style Sheets}}

\newacronym{sass}{SASS}{\textit{Syntactically Awesome Style Sheets}}

\newacronym{js}{JS}{\textit{JavaScript}}

\newacronym{ts}{TS}{\textit{TypeScript}}

\newacronym{nosql}{NoSQL}{No SQL \textemdash \textit{Not Only SQL}}

\newacronym{nvm}{NVM}{\textit{Node Version Manager}}

\newacronym{npm}{NPM}{\textit{Node Package Manager}}

\newacronym{jwt}{JWT}{\textit{JSON Web Token}}

\newacronym{ui}{UI}{\textit{User Interface}}

\newacronym{ux}{UX}{\textit{User Experience}}

\newacronym{http}{HTTP}{\textit{HyperText Transfer Protocol}}

\newacronym{cdn}{CDN}{\textit{Content Delivery Network}}

\newacronym{cms}{CMS}{\textit{Content Management System}}

\newacronym{crud}{CRUD}{\textit{Create, Read, Update, Delete}}

\newacronym{dom}{DOM}{\textit{Document Object Model}}

\newacronym{mvc}{MVC}{\textit{Model-View-Controller}}

\newacronym{rest}{REST}{\textit{Representational State Transfer}}

\newacronym{api}{API}{\textit{Application Programming Interface}}

\newacronym{url}{URL}{\textit{Uniform Resource Locator}}

\newacronym{db}{DB}{\textit{Database}}

\newacronym{ci}{CI}{\textit{Continuous Integration}}

\newacronym{cd}{CD}{\textit{Continuous Delivery}}

\newacronym{ide}{IDE}{\textit{Integrated Development Environment}}

\newacronym{svg}{SVG}{\textit{Scalable Vector Graphics}}

\newacronym{cors}{CORS}{\textit{Cross-Origin Resource Sharing}}

\newacronym{srp}{SRP}{\textit{Single Responsibility Principle}}

\newacronym{web}{Web}{\textit{World Wide Web}}

% ==> Glossary <== %

\newglossaryentry{roles}
{
    name=Roles,
    description={Forma de distinguir os diversos tipos de utilizadores de uma aplicação, contendo como tal diferentes tipos de permissões e ações possíveis de realizar}
}

\newglossaryentry{responsive}
{
    name=\textit{Responsive} / Responsivo,
    description={Conjunto de técnicas aplicadas a um \textit{layout} de forma a este se adaptar a qualquer tamanho de ecrã, independentemente do dispositivo}
}

\newglossaryentry{layout}
{
    name=Layout,
	description={\textit{Forma como são organizadas ou distribuídas as diferentes partes de algo: layout de armazém, layout do teclado.}, por \href{https://www.lexico.pt/layout/}{Lexico}}
}

\newglossaryentry{mockups}
{
    name=Mockups,
	description={prótotipo de um projeto ou dispositivo, tendo como principal objetivo representar as principais funcionalidades do projeto/dispositivo. Utilizado frequentemente em projetos de desenvolvimento web para obter \textit{feedback} do cliente}
}

\newglossaryentry{frontend}
{
    name=\textit{Front-end},
	description={parte vocacionada ao utilizador final, focada na \textit{interface} visualizada, bem como a interação com o sistema. Essencialmente são usadas as linguagens/tecnologias \textbf{HTML}, \textbf{CSS} e \textbf{JavaScript}}
}

\newglossaryentry{backend}
{
    name=\textit{Back-end},
	description={parte vocacionada na parte de implementação contento todas as regras de negócio, não contendo \textit{interface}. Nesta componente podem ser utilizadas linguagens como:
	\begin{itemize}
		\item C\#;
		\item PHP;
		\item Java;
		\item Python;
		\item ...
	\end{itemize}}
}

\newglossaryentry{lazyloading}{
	name={\textit{Lazy Loading}},
	description={Consiste na técnica de adiar o carregamento de determinado componente ou class até este ser necessário.}
}

\newglossaryentry{sprite}{
	name={\textit{Sprite}},
	description={Consiste numa imagem que contêm múltiplas imagens, bastante utilizado para armazenar todos os ícones de uma aplicação num único ficheiro.}
}

% TODO: complete definition
\newglossaryentry{packages}{
	name={\textit{Packages}},
	description={São nada mais nada menos do que módulos que podem ser utilizados em todo o projeto, como por exemplo ...}
}

% TODO: complete definition
\newglossaryentry{scripts}{
	name={\textit{Scripts}},
	description={TODO}
}

% TODO: complete definition
\newglossaryentry{template}{
	name={\textit{Template}},
	description={TODO}
}